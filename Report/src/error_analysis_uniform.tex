\subsection{Error and convergence}
Because we don't know the analytic solution we adopt the \emph{Manufactured Solution} strategy to check the convergence rate of the method : 
\begin{enumerate}
\item We choose $q(x,t)\in \RR^3$ as smooth as possible to be the solution of equations (\ref{eq:EulerGasEquation}). 
\item We put $q$ in the Euler's equations and get, in general, a non zeros result (otherwise, it will mean we have an analytical solution, which is in general not possible). Let's call the rest $\mathcal{S}(x,t)$ : 
$$
\frac{\partial q}{\partial t} + \frac{\partial f(q)}{\partial x} = \mathcal{S}
$$
\item We now consider the modified problem : 
\begin{boxeq}
\begin{split}
\centering
&\frac{\partial U}{\partial t} + \frac{\partial f(U)}{\partial x} = \mathcal{S}\\
&U(x,0)=q(x,0)\\
\text{Plus boundary}&\text{ conditions to be discusses bellow}
\end{split}
\end{boxeq}
For which we know the solution to be $q$.
\item We apply our method to this problem. It requires minor modification in the code to add the source term.
\item We can now compare the numerical solution to the analytical one. And compute the convergence rate.
\end{enumerate}
A common choice in this approach is to choose periodic boundary conditions in keep smooth solutions (which is not systemically the case with wall conditions). Two choice of $q$ are presented bellow in order to be certain that the method is of order $2$.

\subsubsection{Case 1}
This case could be found in \cite{zhu2016new} (example 3.3).\\
For this first case we want the solution to be : 
$$
q= \left\{
\begin{array}{ccl}
\rho (x,t) &=& 1+0.2\sin (2\pi (x-t))\\
u(x,t) &=&1\\
P(x,t) &=& 1\\
\end{array}
\right.
$$ 
The solution have to be periodic that's why the choice of the $\sin$ function is natural to get both smooth and periodic properties.
Then a little computation using the equation of state gives the solution for the energy : 
$$
E(x,t) = \frac{\gamma +1}{2(\gamma -1)} +0.1\sin (2\pi (x-t))
$$
Hence the initial conditions are the following : 
\begin{boxeq}
\begin{split}
\rho(x,0) &= 1 + 0.2\sin (2\pi x) \quad \forall x\in \Omega\\
u(x,0) &= 1 \quad \forall x\in \Omega\\
P(x,0)&= 1 \quad \forall x\in \Omega\\
\end{split}
\label{eq:errorCase1}
\end{boxeq}
In this particular case, the source term is identically zero :
$$
\mathcal{S}(x,t)=\begin{pmatrix}
0\\
0\\
0
\end{pmatrix}
$$ 
Thus, here we got an analytical solution to the problem (which enough remarkable).\\
We make computation for $100,\ 200,\ 300,\ 400,\ 600,\ 800,\ 1200,\ 1600,\ 2400,\ 3200,\ 4000$ $6400,\ 9600,\ 12800,\ 19200,\ 25600,\ 38400,\ 51200$
 nodes and up to $1000$ iterations. Hence we are able to compare the exact solution for the density and the numerical approximation. Figures \ref{fig:errorCase1L1} and \ref{fig:errorCase1Linf} show the $\log$ of the error for the $L^1$ and $L^{\infty}$ norms against the $\log$ of the number of nodes. \\

\begin{figure}[!h]
\captionsetup{justification=centering}
\begin{minipage}{0.45\linewidth}
\centering
\includegraphics[width=\linewidth]{{"../Results/Error/Case 1/uniform_error_L1"}.png}
\caption{\label{fig:errorCase1L1}Convergence rate for manufactured solution \# 1 in $L^1$ norm}
\end{minipage}
\begin{minipage}{0.45\linewidth}
\centering
\includegraphics[width=\linewidth]{{"../Results/Error/Case 1/uniform_error_LInf"}.png}
\caption{\label{fig:errorCase1Linf}Convergence rate for manufactured solution \# 1 in $L^{\infty}$ norm}
\end{minipage}
\end{figure}

Where the $L^1$ and $L^{\infty}$ norms are the following : 
$$
\|U^{ex} - U^{a}\|_1 = \sum_{j=1}^N |U^{ex}(x_j)-U^a_j| \qquad \|U^{ex} - U^{a}\|_{\infty} = \max_{j=1..N} |U^{ex}(x_j)-U^a_j|
$$

The $\log error$ perfectly fit a $-2$ slope line. Hence as we hoped, the method seems to be of order $2$.

\subsubsection{Case 2}
The idea behind this case comes mainly from \cite{gassner2014kinetic}\\
The second case will comfort us in the fact tat the method is of order 2. We want the solution to be : 
$$
q= \left\{
\begin{array}{ccl}
\rho (x,t) &= &2+0.1\sin (2\pi (x-t)) \\
u(x,t) &=& 1\\
E(x,t)& =& 2+0.1\cos (2\pi (x-t))
\end{array}
\right.
$$

Then, the initial conditions are the following :
\begin{boxeq}
\begin{split}
u(x,0) &= 1 \quad \forall x\in \Omega\\
\rho(x,0) &= 2 + 0.1\sin (2\pi x) \quad \forall x\in \Omega\\
P(x,0)&= \frac{\gamma -1}{20}\left(20+2\cos (2\pi x)-sin(2\pi x)\right) \quad \forall x\in \Omega\\
\end{split}
\label{eq:errorCase2}
\end{boxeq}

And the source term is given by : 
\begin{boxeq}
\mathcal{S}(x,t) =
(1-\gamma)\pi (2\rho (x,t) + E(x,t) -6)
\begin{pmatrix}
0\\
1\\
1\\
\end{pmatrix}
\label{eq:sourceCase2}
\end{boxeq}

After modifications of the code to handle the source term, we again make computation for $100,\ 200,\ 300,\ 400,\ 600,\ 800,\ 1200,\ 1600,\ 2400,\ 3200,\ 4000$ $6400,\ 9600,\ 12800,\ 19200,\ 25600,\ 38400,\ 51200$ nodes with step of $100$ nodes up to $1000$ iterations. Hence we are able to compare the exact solution and the numerical approximation. Figures \ref{fig:errorCase2L1} and \ref{fig:errorCase2Linf} shows the $\log$ of the error against the $\log$ of the number of nodes. \\
\begin{figure}[!h]
\captionsetup{justification=centering}
\begin{minipage}{0.45\linewidth}
\centering
\includegraphics[width=\linewidth]{{"../Results/Error/Case 2/uniform_error_L1"}.png}
\caption{\label{fig:errorCase2L1}Convergence rate for manufactured solution \# 2 in $L^1$ norm}
\end{minipage}
\begin{minipage}{0.45\linewidth}
\centering
\includegraphics[width=\linewidth]{{"../Results/Error/Case 2/uniform_error_LInf"}.png}
\caption{\label{fig:errorCase2Linf}Convergence rate for manufactured solution \# 2 in $L^{\infty}$ norm}
\end{minipage}
\end{figure}

Again the convergence rate is really near $2$. We can conclude that the method it of order $2$ as announced.\\

Having satisfying ourselves with the convergence rate we can now observe the method working on several typical example. For each example, I will try to give a real-life situation where the phenomenon describe could occur.

